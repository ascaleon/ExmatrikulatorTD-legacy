%%% Definitionen

\newglossaryentry{anwendungsfall}{%
name={Anwendungsfall},%
description={Spezifiziert eine Menge von Aktionen, die ein System ausführen muss, damit ein Resultat stattfindet, welches für mindestens einen Akteur von Bedeutung ist.}}

\newglossaryentry{Apache Derby}{%
name={Anwendungsfall},%
description={Eine frei verwendbare relationale Datenbank, die eine Java-Applikation \enquote{eingebettet}, d.\,h. ohne Nutzung einer externen Anwendung benutzt werden kann.}}
 
\newglossaryentry{Apache Maven}{%
name={Apache Maven},%
description={Ein Programm zur automatischen Durchführung des
Erstellungsprozesses von Java-Programmen und der Verwaltung von deren Abhängigkeiten.}}

\newglossaryentry{Gradle}{%
name={Gradle},%
description={Ein Programm zur automatischen Durchführung des Erstellungsprozesses von Java-Programmen und der Verwaltung von deren Abhängigkeiten.}}

\newglossaryentry{Einflussfaktor}{%
name={Einflussfaktor},%
description={Ein Faktor, der die Arbeit an der Architektur auf beliebige Art und Weise beeinflusst.}}

\newglossaryentry{Entwurfsmuster}{%
name={Entwurfsmuster},%
description={Musterlösung für ein Problem, welches bei Implementierungen immer wieder auftaucht}}

\newglossaryentry{Framework}{%
name={Framework},%
description={Programmiergerüst, welches den Rahmen der Anwendung bildet. Es umfasst Bibliotheken und Komponenten}}

\newglossaryentry{Java}{%
name={Java},%
description={Eine objektorientierte Programmiersprache.}}

\newglossaryentry{Komponentendiagramm}{%
name={Komponentendiagramm},%
description={Strukturdiagramm der UML, stellt Komponenten
und deren Schnittstellen dar.}}

\newglossaryentry{Turm}{%
name={Turm},%
description={Objekt auf dem Spielfeld, das Gegner angreift und ihnen Schaden zufügt, bis sie zerstört sind. Kann von der Spielerin auf dem Spielfeld gegen das Zahlen von Geld platziert und aufgerüstet werden. Hat je nach Turm-Art einen speziellen Angriffstyp, Spezialangriffe und/oder einen Auraeffekt, der einen passiven Effekt auf alle Türme/Gegner in der Umgebung hat.}}

\newglossaryentry{Gegner}{%
name={Gegner},%
description={Feindliche Einheit, die auf einem vorbestimmten Pfad versucht, ein Zielfeld zu erreichen. Besitzt je nach Gegner-Typ eine bestimmte Menge Lebenspunkte, einen Rüstungstyp, }}

% \newglossaryentry{Memento}{%
% name={Memento},%
% description={Ein Verhaltensmuster, welcher der Erfassung und Externalisierung des internen Zustands von Objekten dient.}}

\newglossaryentry{Paketdiagramm}{%
name={Paketdiagramm},%
description={Strukturdiagramm der UML, stellt die Verbindung
zwischen Paketen, Paketimports bzw. Verschmelzungen und deren Abhängigkeiten dar.}}

\newglossaryentry{Problemkarte}{%
name={Problemkarte},%
description={Beschreibt ein Problem im Zusammenhang von zuvor zusammengestellten Einflussfaktoren und führen Lösungen bzw. entsprechende Strategien zur Lösung des Problems an}}

\newglossaryentry{Sequenzdiagramm}{%
name={Sequenzdiagramm},%
description={Strukturdiagramm der UML, stellt den Austausch
von Nachrichten zwischen Objekten mittels einer Lebenslinie dar.}}


\newglossaryentry{Softwareachitektur}{%
name={Softwarearchitektur},%
description={Struktur und der Beziehung der einzelnen Bestandteile einer Software zueinander}}

\newglossaryentry{.jar-Datei}{%
name={JAR-Datei},%
description={Von einer JVM ausführbare Datei, die aus Java-Code übersetzt wurde}}

\newglossaryentry{Thread}{%
name={Thread},%
description={Teil eines Prozesses, der parallel zu anderen Threads ausgeführt werden kann}}

\newglossaryentry{Datenbank}{%
name={(Relationale) Datenbank},%
description={System zur Verwaltung von Daten, die als Einträge in Tabellen modelliert werden}}

% \newglossaryentry{Log}{%
% name={Log},%
% description={Text-Datei oder Datenbank, in der die Aktionen Nutzer*innen protokolliert werden, damit etwa Auswerter*innen anschließend das Nutzungsverhalten nachvollziehen und analysieren können}}

% \newglossaryentry{Command}{%
% name={Command-Pattern},%
% description={Entwurfsmuster, bei dem Nutzer*innen-Aktionen als Kommando abstrahiert werden, um Aktionen abspeichern zu können und ggf. rückgängig machen bzw. wiederherstellen zu können}}

% \newglossaryentry{Umfang}{%
% name={Umfang},%
% description={Definiert als \enquote{Anzahl wertbarer Symptome + Anzahl wertbarer Relationen} \cite{Anforderungen2018}}}

% \newglossaryentry{Vernetzungsindex}{%
% name={Vernetzungsindex},%
% description={Definiert als \enquote{2 x Anzahl wertbarer Relationen / Anzahl wertbarer Symptome} \cite{Anforderungen2018}}}

% \newglossaryentry{Strukturindex}{%
% name={Strukturindex},%
% description={Definiert als \enquote{Summe aller Pfeilketten, Verzweigungen und Kreisläufe / Anzahl wertbarer Symptome} \cite{Anforderungen2018}}}



%%% Abkürzungen

% \newacronym{DAO}{DAO}{\textit{Data Access Object}. Entwurfsmuster, das den Zugriff auf unterschiedliche Arten von Datenquellen so kapselt, dass die angesprochene Datenquelle ausgetauscht werden kann, ohne dass }

\newacronym{GUI}{GUI}{\textit{Graphical User Interface}, grafische Schnittstelle, über die ein Mensch mit einer Software interagieren kann}

% \newacronym{MVC}{MVC}{\textit{Model-View-Controller}, Entwurfsmuster zur Strukturierung von Software, bei dem die Benutzerschnittstelle, Programmlogik und Datenhaltung in separate Komponenten gekapselt werden}

\newacronym{IDE}{IDE}{\textit{Integrated Delevopment Environment}, eine integrierte Entwicklungsumgebung, die Nutzer*innen bei allen Belangen der Softwareentwicklung unterstützt, sei es beim Schreiben von Programmcode, beim Testen oder der Versionskontrolle}

\newacronym{SWP}{SWP}{\textit{Softwareprojekt}, eine in zwei Teile geteilte Veranstaltung im Bachelorstudiengang Informatik an der Universität Bremen, in der Studierende eine eigene Software entwickeln müssen}

% \newacronym{JSON}{JSON}{\textit{JavaScript Object Notation}, tabellarisch organisiertes Dateiformat, in dem Spalten durch Kommata repräsentiert werden}

% \newacronym{CSV}{CSV}{\textit{Comma Seperated Values}, tabellarisch organisiertes Dateiformat, in dem Spalten durch Kommata repräsentiert werden}

\newacronym{JVM}{JVM}{\textit{Java Virtual Machine}, virtuelle Maschine, die es ermöglicht, Java-Code plattformübergreifend auszuführen}

\newacronym{SQL}{SQL}{\textit{Structured Query Language}, Sprache zum Definieren, Modifizieren und Abfragen von Datenbankstrukturen in einer relationalen Datenbank.}

\newacronym{UML}{UML}{\textit{Unified Modeling Language}, grafische Modellierungssprache zur Spezifikation, Visualisierung, Konstruktion und Dokumentation von Modellen für Softwaresysteme.}

% \newacronym{WBGU}{WBGU}{\textit{Wissenschaftliche Beirat der Bundesregierung für Globale Umweltveränderung}, Expert*innengremium, das die Bundesregierung bei umweltpolitischen Fragestellungen berät und auf wissenschaftlicher Grundlage Handlungsempfehlungen gibt}

% \newacronym{GLX}{GLX}{\textit{Graph eXchange Language}, auf XML basierendes, standardisiertes Austauschformat für Graphen}

 \newacronym{DAO}{DAO}{\textit{Database Access Object}, Entwurfsmuster, das die Definition einer Schnittstelle vorsieht, über die Java-Objekte in einer relationalen Datenbank gespeichert, aktualisiert und abgerufen werden können. }

% \newacronym{XML}{XML}{\textit{Extensible Markup Language}, flexibel erweiterbares Textformat zum Austausch von Daten (vor allem über das Internet)}

% \newacronym{PDF}{PDF}{\textit{Portable Document Format}, plattformübergreifendes Austauschformat für Dokumente}

\newacronym{JPA}{JPA}{\textit{Java Persistence API}, Framework bzw. Schnittstelle zur Modellierung von Java-Objekten als Datenbankeinträge}

% \newacronym{Java EE}{Java EE}{\textit{Java Enterprise Edition}, Framework, das vor allem zur Realisierung von Webanwendungen dient}

%Time-to-Market Beschreibt die Dauer, angefangen bei der Produktentwicklung, bis hin zur Auslieferung am Markt/Kunden.
%Transaktionssichere Datenbank Eine Datenbank, die eigentlich getrennte Schreiboperationen zu einer sog. Transaktion zusammenfassen und somit effektiv atomar ausführen kann, sodass entweder alle in der Transaktion gekapselten Änderungen durchgeführt werden (oder keine), wobei ggf. bereits zum Teil geschriebene Daten automatisch auf den Ausgangszustand versetzt werden, wenn die Transaktion fehlschlägt.

